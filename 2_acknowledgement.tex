\chapter*{Remerciements}\markboth{REMERCIEMENTS}{}

\thispagestyle{empty}
\justify
\quad 
Je voudrais en tout premier lieu exprimer toute ma reconnaissance à mes deux superviseurs Patrick Llerena et Stefano Bianchini pour m'avoir encadré. Leur grande rigueur scientifique, leurs conseils, mais aussi leur confiance, ont grandement contribué à la réussite de cette thèse. Je voudrais également remercier Madame Sandrine Wolff, Directrice de Recherche à l’INRAE, Madame la Professeure Fabiana vinsentin, Monsieur Pablo d'Este, Directeur de Recherche au CNRS, Monsieur Jean-Marc Deltorne, Chargé de Recherche à l’INRAE pour m’avoir fait l’honneur de composer mon jury. Mes remerciements vont aussi à Monsieur le Professeur Nicolas Lachiche et Robin Cowan qui ont accepté de faire partie de mon Comité de Suivi dès ma seconde année de thèse. Je remercie également les coauteurs ayant collaboré aux travaux de cette thèse. Merci à Stefano Bianchini, Pierre Pelletier, Moritz Müller, Roman Jurowetski et Daniel Hain en ésperant continuer à collaborer. Je remercie le Bureau d’Économie Théorique et Appliquée (BETA) et l’École doctorale Augustin Cournot pour avoir mis à ma disposition l’ensemble des moyens intellectuels pour la réalisation de ce travail.

Je tiens à remercier particulièrement certains doctorant.e.s et docteur.e.s qui m’ont accompagné durant ce long périple. En premier une personne sans qui je ne me serais jamais lancé dans une thèse et sans qui celle-ci serait probablement pas fini: Pierre. Merci pour tout, je suivrais ta carrière d'économiste avec intérêt!! Merci également à Eva, je sais que tu t'attends à un paragraphe entier sur ta personne mais même en utilisant uniquement des mots-clés des délires qu'on a eu le paragraphe serait plus long que ma thèse (j'abuse à peine). Merci à la team Smash Nico et Heman, quand vous voulez pour me prendre un Bo3. Merci à la team de runner Théo, Louis et particulièrement Sarah qui à du me carry sur le Ekiden alors que j'étais au bout de ma life. Merci à mon bureau beaucoup trop rempli; Agathe et les récits de sa vie qui pourrait probablement être une série Netflix; Guillaume sa bouture et ses grains de cafés; Emilien et son tabassage de bureau; Shengxi pour ses cadeaux; mais aussi Lucas, Diletta et Mathilde. Merci aux autres doctorants: Anne-Gaëlle pour les soirées Doctor Who; Jérome et son début de carrière de twittos; Romane et sa motivation sans faille pour le clubbing; Kenza et les dingueries qu'elle peut lacher; Alexandre et ses passions jeux vidéo/Karaté/stata; Vincent et sa non possession de carte pro.  

Je n’oublie également pas mes ami.e.s hors de murs du BETA qui ont su me laisser faire mon délire sans me poser trop de questions. Tanguy qui en l'espace de ma thèse à accompli beacoup trop de choses et qui sait probablement mieux coder que moi maintenant. Léo qui à accepté de se faire bolosse bien trop souvent à smash mais aux jeux de société aussi. Merci Aurore pour tous les cinés, les moscow mule, l'eau courante et autres inside joke que toi seule connait. Merci Nico et toutes les heures passé en co-op pendant le Covid. Merci à Claire, c'est bien parce que c'est toi que j'accepte de partager mon anniversaire (et mention spécial à Lucy). Merci Matt pour m'avoir supporté depuis tellement d'années (Ou alors c'est moi qui t'ai supporté ?). Merci à Phil et Ana qui eux aussi m'ont soutenu depuis bien des années et sur qui je sais que je pourrais encore compter. Merci à Arthur pour sa transparence, bienveillance et ses refs perchés. Merci Hugo pour ces discussions jusqu'a pas d'heure. Merci à Dresch, tu reviens quand tu veux pour parler de blockchain. Merci les frères grillet pour m'avoir soutenu dans le début chaotique de la thèse. 

Je tiens enfin à exprimer ma reconnaissance éternelle à ma famille pour leur soutien indéfectible depuis le début de mes années d’études. Merci mes parents pour leur soutien malgrés le flou dans lequel je les plongeais. Merci à mon frère sans qui j'aurais probablement pas fait autant d'années d'études. Merci à ma Soeur  

Merci à vous tous, sans vous je n’aurais sans doute jamais pu réaliser cette thèse. 
\newpage