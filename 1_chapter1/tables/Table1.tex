% For tables use
\begin{table}
	\begin{threeparttable}
% table caption is above the table
	\caption{World's largest science countries.}
	\label{tab:WorldScienceOutput}       % Give a unique label
% For LaTeX tables use
		\begin{tabular}{lcccc}
\hline\noalign{\smallskip}
 & 1973 & 1981-1994 & 1997-2001 & 2008-2018 \\
 & (Frame et al., 1977) & (May, 1997) & (King, 2004) & (Allik, 2020) \\ 
\noalign{\smallskip}\hline\noalign{\smallskip}
1 & US (38.2) & US (34.6) & US (34.6) & US (19.9) \\
2 & UK (9.2) & UK (8.0)	 & UK (8.5) & China (11.7) \\
3 & USSR (9.0) & Japan	(7.3) & Japan (8.0) & UK (6.0) \\
4 & West Germany	(6.0)	& Germany (7.0) & Germany (7.4) & Germany (5.3) \\
5 & France (5.5) & France (5.2) & France (5.6) & Japan (4.1) \\
6 & Japan (5.2) & Canada (4.5) & Canada (4.6) & France (3.7) \\
7 & Canada (4.4) & Italy	(2.7) & Italy (3.3) & Canada (3.3) \\
8 & India (2.5)	& India (2.4) & Russia (3.3) & Italy (3.2) \\
9 & Australia (1.9) & Australia	(2.1) & India (2.8) & India (2.8) \\
10 & Italy (1.7) & Netherlands (2.0) & Netherlands (2.3) & Australia (2.8) \\
\noalign{\smallskip}\hline\noalign{\smallskip}
Total share & 84\% & 76\% & 80\% & 63\% \\
\noalign{\smallskip}\hline
		\end{tabular}
		\begin{tablenotes}
			\small
			\item Frame et al. (1977) covers 2,300 journals indexed by SCI, May (1997) covers 4,000 journals indexed by ISI, King (2004) covers 8,000 journals indexed by ISI, Alik (2020) covers 12,000 journals indexed by ESI.
		\end{tablenotes}
	\end{threeparttable}
\end{table}